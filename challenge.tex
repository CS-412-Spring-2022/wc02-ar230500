\documentclass[a4paper]{exam}

\usepackage{amsmath,amsfonts}
\usepackage{geometry}
\usepackage{graphicx}
\usepackage{float}

\printanswers

\title{Weekly Challenge 02: Asymptotic Bounds}
\author{CS 412 Algorithms: Design and Analysis}
\date{Spring 2022}

\begin{document}
\maketitle

\begin{questions}
  
\question We are given the following relations among the functions, $f, g, h$, and $i$.
  \begin{align*}
    f(n) = O(g(n))\\
    f(n) = \Omega(h(n))\\
    f(n) = \Theta(i(n))
  \end{align*}




  \noindent\textbf{TASK}: Visualize these relations in a single plot by making appropriate choices for $f, g, h$, and $i$. Make sure that the plot clearly indicates the constants. \\
  \noindent\textbf{TASK}: Include the plot below.
  
  \noindent\textbf{TASK}: Indicate below the functions that you have chosen for $f, g, h$, and $i$ and explain below how the indicated constants fulfill the corresponding definitions. Feel free to refer to the visualization in your explanation.
  
  \noindent\textbf{TASK}:  Also share your visualization as a comment on the \textit{Week 02 Challenge} post in the course group on Yammer.

  \begin{solution}
    Following are my choices:
    \[ f(n) = n\cdot \text{log}(n) \]
    \[ g(n) = 2^n \]
    \[ h(n) = \text{log}(n) \]
    \[ i(n) = 2n\cdot \text{log}(n) \]
    \[ i(n) = 1/2n\cdot \text{log}(n) \]
    \begin{center}
        \textbf{Task 1: Graph}
    \end{center}
    Following is the graph of these functions:
    \begin{figure}[H]
        \centering
        \includegraphics[width=10cm, height=10cm]{Figure2.png}
        \caption{Time Complexity Analysis}
        \label{fig:my_label}
    \end{figure}
    \begin{center}
    \textbf{Task 2: Explanation}
\end{center}
\underline{$f(n) = O(g(n))$}:
\[ n\cdot \text{log}(n) = O(2^n) \]
$g(n)$ is the upper bound of $f(n)$ as we can see in Figure 1 that for the same values of $n$, $g(n)$ increases rapidly than $f(n)$ and for larger values of $n$, $g(n)$ will get even larger. log$(n)$ in $f(n)$ grows very slowly and $n$ in $f(n)$ does not grow larger than $2^n$ (an exponential function).\\ \\
\underline{$f(n) = \Omega (h(n))$}:
\[ n\cdot \text{log}(n) = \Omega (\text{log}(n)) \]
As we can see in Figure 1, the function $h(n)$ grows slower than the function $f(n)$ due to which it is a lower bound to the function $h(n)$. $f(n)$ grows faster due to the factor $n$ that makes it increase rapidly than $h(n)$. \\ \\
\underline{$f(n) = \Theta(i(n))$}:
\[ n\cdot \text{log}(n) = \Theta(2n\text{log}(n)) \]
\[ n\cdot \text{log}(n) = \Theta(1/2n\text{log}(n)) \]
The highest power of $i(n)$ and $f(n)$ is the same due to the constants 1/2 and 2 in the function $i(n)$ create both lower and upper bounds for the function $f(n)$. It means that for very high values of $n$, $i(n)$ and $f(n)$ are going to increase at the same rate. 

  \end{solution}

\end{questions}
\end{document}

%%% Local Variables:
%%% mode: latex
%%% TeX-master: t
%%% End:
